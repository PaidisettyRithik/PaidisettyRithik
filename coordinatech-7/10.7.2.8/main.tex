\documentclass[12pt]{article}
\usepackage{amsmath}
\newcommand{\myvec}[1]{\ensuremath{\begin{pmatrix}#1\end{pmatrix}}}
\newcommand{\mydet}[1]{\ensuremath{\begin{vmatrix}#1\end{vmatrix}}}
\newcommand{\solution}{\noindent \textbf{Solution: }}
\providecommand{\brak}[1]{\ensuremath{\left(#1\right)}}
\providecommand{\norm}[1]{\left\lVert#1\right\rVert}
\let\vec\mathbf

\title{Coordinate-Geomentry}
\author{Paidisetty Rithik(paidisettyrithik@sriprakashschools.com)}

\begin{document}
\maketitle
\section*{10$^{th}$ Maths - Chapter 7}
This is Problem-8 from Exercise 7.2
\begin{enumerate}
\item if A and B are\myvec{-2\\-2} and \myvec{2\\-4},respectively,find the coordinates of P such that
AP=$\frac{3}{7}$ AB and P lies on the segment AB \\
\end{enumerate}
\solution \\
Given,\\
A=\myvec{-2\\-2}, B=\myvec{2\\-4},
$m_1:m_2=3:4$
\begin{align}
P=\frac{m_1B+m_2A}{m_1+m_2}\\
P=\frac{3\myvec{2\\-4}+4\myvec{-2\\-2}}{3+4}\\
P=\frac{\myvec{6-8\\-12-8}}{3+4}
\end{align}
\begin{align}
P(x)=\frac{6-8}{3+4}
=\frac{-2}{7}\\
P(y)=\frac{-12-8}{3+4}
=\frac{-20}{7}
\end{align}
Therefore the coordinates of P are $\dfrac{-2}{7}$,$\dfrac{-20}{7}$



\end{document}