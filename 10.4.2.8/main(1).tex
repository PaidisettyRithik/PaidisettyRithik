\documentclass[12pt]{article}
\usepackage{amsmath}
\newcommand{\myvec}[1]{\ensuremath{\begin{pmatrix}#1\end{pmatrix}}}
\newcommand{\mydet}[1]{\ensuremath{\begin{vmatrix}#1\end{vmatrix}}}
\newcommand{\solution}{\noindent \textbf{Solution: }}
\providecommand{\brak}[1]{\ensuremath{\left(#1\right)}}
\providecommand{\norm}[1]{\left\lVert#1\right\rVert}
\let\vec\mathbf

\title{Quadratic-Equations}
\author{Paidisetty Rithik(paidisettyrithik@sriprakashschools.com)}

\begin{document}
\maketitle
\section*{10$^{th}$ Maths - Chapter 4}
This is Problem-3 from Exercise 4.2
\begin{enumerate}
\item Find two numbers whose sum is 27 and produict is 182\\
\end{enumerate}
\solution \\
let the first number be x,
therefore the second will be 'x-27'\\
Given:\\
\begin{align}
(x)(x-27) &= 182\\
x^2-27x &= 182\\
x^2-27x-182 &= 0
\end{align}
Using formula method,first solution is:\\
\begin{align}
x_1 &= \frac{-b+\sqrt{b^2-4ac}}{2a}\\
x_1 &= \frac{-(-27)+\sqrt{(-27)^2-4(1)(-182)}}{2(1)}\\
x_1 &= \frac{27+\sqrt{729+728}}{2}\\
x_1 &= \frac{27+\sqrt{1457}}{2}
\end{align}
the second solution is:\\
\begin{align}
x_2 &= \frac{-b-\sqrt{b^2-4ac}}{2a}\\
x_2 &= \frac{-(-27)-\sqrt{1457}}{2}\\
x_2 &= \frac{27-\sqrt{1457}}{2}
\end{align}



\end{document}